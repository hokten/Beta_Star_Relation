\documentclass[a4paper,12pt]{article}
\usepackage[top=3cm,bottom=2cm,left=3cm,right=2cm]{geometry}
%\usepackage{ucs}
\usepackage[utf8]{inputenc}
\usepackage[T1]{fontenc} 
\usepackage{amsmath}
\usepackage{amsfonts}
\usepackage{amsthm}
\usepackage{amssymb}
\usepackage{setspace}
\usepackage{mathptmx}
\usepackage{titlesec}
\usepackage{enumitem}

\numberwithin{equation}{section}
%\usepackage[lflt]{floatflt}
\renewcommand{\labelenumi}{(\roman{enumi})}
%\hoffset	= 0.96cm
%\textheight	=614pt
%\footskip	=-40mm
%\marginparwidth = 0mm
%\evensidemargin	=0mm
%\marginparsep	= 0mm
%\topmargin	=30mm	
%\oddsidemargin	=-10.mm		
%\evensidemargin	=0mm		
%\headheight	=0.mm
%\headsep	=0.mm
%\textheight	=220.mm
%\textwidth	=150.mm
\title{Ders Notları}
\date{}
\def\chaptername{Bölüm}
\newtheoremstyle{italik}{}{}{\normalfont}{}{\bfseries}{.\ }{ }{} % basic definitions, use roman font: 
\theoremstyle{italik}
\newtheorem{ornek}{Örnek}[part]
\newtheorem{teorem}{Teorem}[section]
\newtheorem{lemma}[teorem]{Lemma}
\newtheorem{sonuc}[teorem]{Corollary}
\newtheorem{onerme}[teorem]{Önerme}
\newtheorem{tanim}[teorem]{Definition}
\newtheorem*{ispat}{Proof}
\newtheorem*{cozum}{Çözüm}

\newtheorem{orneks}{Örnek}[part]
\newtheorem{teorems}{Teorem}[section]
\newtheorem{onermes}[teorems]{Önerme}
\newtheorem{lemmas}[teorems]{Yardımcı Teorem}
\newtheorem{sonucs}[teorems]{Sonuç}
\newtheorem{tanims}{Tanım}[section]
\def\contentsname{İÇİNDEKİLER}

\titleformat{\section}{
\titlerule \vspace{.8ex}% 
\large\bfseries}
{\thesection.}{.5em}{}

\begin{document}
\section{Introduction}
Throughout this paper L denotes a lattice.

\begin{tanim} \label{1}
  Let $ a,b $ elements of $ L $. We define a relation $ \beta_* $ on the elements of $ L $ by 
  $ a \beta_* b $ if and only if for each $ t \in L $ such that $ a \vee t = 1 $ then $ b \vee t = 1 $ and for each 
  $ k \in L $ such that $ b \vee k = 1 $ then $ a \vee k = 1 $.
\end{tanim}

\begin{lemma} \label{2}
  $ \beta_* $ is equivalence relation.
\end{lemma}
\begin{ispat}
  The reflexive and symmetric properties are clear. For transitivity, assume $ a \beta_*b $ and $ b \beta_* c $. Let 
  $ t $ be an element of $ L $ such that $ a \vee t = 1 $. Since $ a \beta_*b $, $ b \vee t = 1 $. So by $ b \beta_* c = 1 $, 
  $ c \vee t = 1 $. Similarly, we can show for each $ k \in L $ such that $ a \vee k = 1 $ then $ a \vee k=1 $. 
  Thus $ a \beta_* c $.

\end{ispat}

\begin{teorem} \label{3}
  Let $ a,b $ be elements of $ L $. 
  \begin{enumerate}
    \item
      $ a \beta_* b $ if and only if $ a \vee c = 1 $ and $ b \vee c = 1 $ for each $ c \in L $ 
      such that $ a \vee b  \vee c = 1 $.
    \item
      $ a \beta_* b $ if and only if $ a \vee b \ll 1/a $ and $ a \vee b \ll 1/b $
  \end{enumerate}
\end{teorem}


\begin{ispat}
  \begin{enumerate}
    \item
      $ ( \Rightarrow ) $
      Let $ a \beta_* b$ and $ c $ be a element of $ L $ such that $ a \vee b \vee c = 1 $. Since $ a \vee ( b \vee c ) = 1 $ and 
      $ a \beta_* b $, $ b \vee ( b \vee c ) = 1 $. Hence $ b \vee c = 1 $. Similarly, $ a \vee c = 1 $. \\
      $( \Leftarrow )$
      Let $ t $ a be element of $ L $ such that $ a \vee t = 1 $. 
      Hence $ a \vee b \vee c = ( a \vee t ) = 1 $. By the hypothesis $ b \vee t = 1 $. 
      Similarly, $ a \vee k = 1 $ for each $ k \in L $ such that $ b \vee k = 1 $. 
      Therefore $ a \beta_* b $.
    \item
      $ ( \Rightarrow ) $
      Let i$ a \beta_* b $ and $ t $ be a element of $ 1/a $ such that $ a \vee b \vee t = 1 $. 
      Hence $ ( a \vee t ) \vee b = 1 $. Since $ a \beta_* b $, $ a \vee t = a \vee a \vee t = 1 $. 
      Since $ a \leq 1 $, $ t=1 $. Therefore $ a \vee b \ll 1/a $. 
      Similarly, we can show $ a \vee b \ll 1/b $. \\
      $( \Leftarrow )$
      Let $ a \vee b \ll 1/a $ , $ a \vee b \ll 1/b $ and $ t \in L $ such that $ a \vee b = 1 $. 
      So $ ( a \vee b ) \vee ( b \vee t ) = a \vee b \vee t = 1 $. Since $ a \vee b \ll 1/b $ and 
      $ b \vee t \in 1/b $, $ b \vee t = 1 $. 
      Similarly, $ a \vee k = 1 $ for each $ k \in L $ such that $ b \vee k = 1 $. 
  \end{enumerate}
\end{ispat}

\begin{teorem} \label{4}
  Let $ a,b $ be elements of $ L $.
  \begin{enumerate}
    \item
      If $ a \ll L $ and $ a \beta_* b $ then $ b \ll L $.
    \item
      All small elements in $ L $ is equilavent for $ \beta_* $ equivalence relation.
  \end{enumerate}
\end{teorem}

\begin{ispat}
  \begin{enumerate}
    \item
      Let $ a \beta_* b $ , $ a \ll L $ and $ t \in L $ such that $ b \vee t = 1 $. 
      So $ a \vee t = 1 $. Since $ a \ll L $, $ t = 1 $. Thus $ b \ll L $. 
    \item
      Let $ a \ll L $ and $ b \ll L $ for $ a,b \in L $. Since $ a \ll L $, if $ a \vee t = 1 $ 
      then $ t = 1 $. Therefore $ b \vee t = 1 $. 
      Similarly, $ b \vee k = 1 $ for each $ k \in L $ such that $ a \vee k = 1 $. 
      Thus $ a \beta_* b $. 
  \end{enumerate}
\end{ispat}

\begin{sonuc} \label{5}
  $ L $ is hollow if and only if all small elements(different form $ 1 $) in $ L $ are equilavent for $ \beta_* $ relation. 
\end{sonuc}
\begin{ispat}
  $ ( \Rightarrow ) $
  Let $ L $ be hollow. Then all elements of $ L $ are small in $ L $. Clear from Theorem \ref{5}. \\
  $ ( \Leftarrow ) $ 
  Let all elements of $ L $ be equilavent to each other. Let us take any element $ t $ in $ L $ 
  such that $ a \vee t = 1 $. By our hypothesis, $ a \beta_* t $. Hence $ t = t \vee t = 1 $. 
  Therefore $ a \ll L $ and so $ L $ is hollow. 
\end{ispat}
\begin{tanim}
  Let $ a, b \in L $ and $ a \leq b $. We say that $ b\  lies\  above\  a $ if, 
  $ b \vee t = 1 $ for any $ t \in L $.
\end{tanim}
\begin{teorem}\label{6}
  Let $ a,b $ be elements of $ L $ such that $ a \leq b $. 
  If $ b $ lies above $ a $ then $ a \beta_* b $. 
\end{teorem}
\begin{ispat}
  Let us say $ a $ lies above $ b $. By definition, for any $ t \in L $ such that $ b \vee t = 1 $, 
  $ a \vee t = 1 $. Moreover, let $ a \vee k = 1 $ for a $ k \in L $. 
  Since $ a \leq b $, $ a \vee b \vee k = 1 $. Therefore $ b \vee k = 1 $ and so $ a \beta_* b $.

\end{ispat}
\begin{lemma}\label{7}
  Let $ a,b,c $ be elements of $ L $.
  If $ a \vee b = 1 $ and $ ( a \wedge b ) \vee c =1 $
  then $ a \vee(b \wedge c)=b \vee ( a \wedge c ) = 1 $.
\end{lemma}

\begin{teorem}\label{8}
  Let $ a,b \in L $. If $ a\beta_* b $ then the following conditions hold.
  \begin{enumerate}[label=(\roman{*}), ref=(\roman{*})]
      \item
        If there are exists supplements of $ a $ and $ b $ then these are same. \label{8.1}
      \item
        If there are exists weak supplements of $ a $ and $ b $ then these are same. \label{8.2}
    \end{enumerate}
\end{teorem}
\begin{ispat}
  \begin{enumerate}
    \item
      Let $ c $ be supplement of $ a $. Then $ a \vee c = 1 $. Since $ a \beta_* b $, $ b \vee c=1 $.
      Let $ d $ be element of $ L $ such that $ d \leq c $ and $ b \vee d = 1 $. Therefore 
      $ a \vee d = 1 $. Since $ c $ is a supplement of $ a $ and $ d \leq c $, $ d = c $. Then 
      $ c $ is also a supplement of $ b $. Similarly, interchanging the roles of $ a $ and $ b $ 
      we can show that each supplement of $ b $ is also supplement of $ a $.
    \item
      Let $ a \beta_* b $ and $ c $ be a weak supplement of $ a $ in $ L $. Therefore $ a \vee c = 1 $ 
      and $ a \wedge c \ll L $. Since $ a \beta_* b $ and $ a \vee c = 1 $, $ b \vee c = 1 $. 
      Let $ t $ be an element of $ L $ such that $ (b \wedge c ) \vee t = 1 $ . By Lemma \ref{7}, 
      we can find $ t = 1 $. Therefore $ b \wedge c \ll L $ and so $ c $ is also weak supplement of $ b $. 
      Similarly, interchanging the roles of $ a $ and $ b $ we can show that each supplement of $ b $ 
      is also supplement of $ a $.
  \end{enumerate}
\end{ispat}
%00000000000000000000000000000000000000000000000000000000000000000000000000000000000000000000000000000000000000000000
%                                                 Corollary 2.12
%00000000000000000000000000000000000000000000000000000000000000000000000000000000000000000000000000000000000000000000




\begin{teorem}\label{9}
  Let $ L $ be a bol supplemented lattice and $ a,b \in L $. If supplements of $ a $ and $ b $ in $ L $ 
  are same then $ a \beta_* b $.
\end{teorem}

\begin{ispat}
  Let $ t $ be an element of $ L $ such that $ a \vee t = 1 $. Since $ L $ is bol supplemented, there exists 
  a supplement of $ a $ in $ L $ such that $ r \leq t $ and $ a \vee r = 1 $. By the hypothesis, 
  $ r $ is also a supplement of $ b $. Then $ b \vee r = 1 $. Since $ r \leq t $, $ b \vee t = 1 $. 
  Similarly, interchanging the roles of $ a $ and $ b $ we can show that $ a \vee k = 1 $ for any 
  element $ k $ of $ L $ such that $ b \vee k = 1 $. Therefore $ a \beta_* b $.
\end{ispat}
%00000000000000000000000000000000000000000000000000000000000000000000000000000000000000000000000000000000000000000000
%                                                 Corollary 2.12
%00000000000000000000000000000000000000000000000000000000000000000000000000000000000000000000000000000000000000000000

\begin{sonuc}\label{10}
  Let $ x,y,c \in L $ such that $ x \leq y $ and $ c $ is a weak supplement of $ x $ in $ L $. Then 
  $ x \beta_* y $ if and only if $ y \wedge c \ll L $.
\end{sonuc}

\begin{ispat}
  $ ( \Rightarrow ) $ \\
  Clear from Theorem \ref{8} \ref{8.2}.
  $ ( \Leftarrow ) $ 
  Since $ x \leq y $, for any element $ t $ of $ L $ such that $ x \vee t = 1 $, $ y \vee t = 1 $. 
  Let $ k \in L $ such that $ y \vee k = 1 $. Since $ c $ is weak supplement of $ x $ in L, 
  $ x \vee c = 1 $ and $ x \wedge c \ll L $. Therefore $ y \wedge ( x \vee c ) = 1 \wedge y $, and so 
  $ y = x \vee ( y \wedge c ) $. Hence $ k \vee x \vee ( y \wedge c ) = 1 $. Since $ y \wedge c \ll L $, 
  $ x \vee k = 1 $. Thus $ x \beta_* y $.
\end{ispat}

%00000000000000000000000000000000000000000000000000000000000000000000000000000000000000000000000000000000000000000000
%                                                 Theorem 2.13
%00000000000000000000000000000000000000000000000000000000000000000000000000000000000000000000000000000000000000000000

\begin{teorem}\label{11}
  Let $ x,y,z,a,b \in L $ such that $ a \oplus b = 1 $ and $ y $ is a supplement of $ x $ in $ L $. Then
  \begin{enumerate}[label=(\roman{*}), ref=(\roman{*})]
    \item
      If $ z \beta_* y $ then $ z / z \wedge x \cong y / y \wedge x $. \label{11.1}
    \item
      If $ x \beta_* b $ then $ z / z \wedge a \cong b / 0 $. \label{11.2}
    \item
      Let $ z \leq b $. Then $ z \beta_* b $ if and only if $ z = b $.\label{11.3}
    \item
      Let $ b \leq z $. Then $ z \beta_* b $ if and only if $ z \wedge a \ll L $.\label{11.4}
  \end{enumerate}
\end{teorem}
\begin{ispat}
  \begin{enumerate}
      \item
        Since $ y $ is a supplement of $ x $ in $ L $ and $ z \beta_* y $, 
        $ x \vee y = x \vee z = 1 $. Since $ z / z  \wedge x \cong z \vee x / x $ and 
        $ y \vee x / x \cong y / y \wedge x $, we can write $ z / z \wedge x \cong y / y \wedge x $. 
        Thus $ z / z \wedge x \cong y \ y \wedge x $.
      \item
        By \ref{11.1}, $ z / z \wedge a \cong b / a \wedge b $. Since $ a \oplus b = 1 $, 
        $ z / z \wedge a \cong b / 0 $.
      \item
        $ ( \Rightarrow ) $ 
        Since $ a \oplus b = 1 $ and $ z \beta_* b $, $ a \vee z = 1 $. Since $ b $ is a supplement of 
        a in $ L $ and $ z \leq b $, $ z = b $. \\
        $ ( \Leftarrow ) $ 
        Clear from reflexive property of $ \beta_* $.
      \item
        $ ( \Rightarrow ) $ 
        Since $ a $ is weak supplement of $ b $ and $ z \beta_* b $, it follows from 
        Theorem \ref{8} \ref{8.2} that $ a $ is also weak supplement of $ z $. Hence $ z \wedge a \ll L $. \\
        $ ( \Leftarrow ) $ 
        It is clear from Corollary \ref{9}.
    \end{enumerate}
\end{ispat}

%00000000000000000000000000000000000000000000000000000000000000000000000000000000000000000000000000000000000000000000
%                                                 Theorem 2.14
%00000000000000000000000000000000000000000000000000000000000000000000000000000000000000000000000000000000000000000000

\begin{teorem}\label{12}
  Let $ L $ be a distributive lattice and $ a,b \in L $. If $ a \oplus b = 1 $ and $ a \beta_* x $ 
  then $ a \leq x $ and $ b \wedge x \ll L $.
\end{teorem}
\begin{ispat}
    Since $ a \oplus b = 1 $ and $ a \beta_* x $, $ x \vee b = 1 $. Hence $ a \wedge ( x \vee b ) = a \wedge 1 = a $. 
    By the modular law, $ ( a \wedge x ) \vee ( a \wedge b ) = a $. Since $ a \wedge b = 0 $, 
    it follows that $ a \wedge x = a $, and so $ a \leq x $. Since $ x \beta_* a $ and $ a \leq x $, 
    it follows from that Theorem \ref{11} \ref{11.4} $ b \wedge x \ll L $.
\end{ispat}

%00000000000000000000000000000000000000000000000000000000000000000000000000000000000000000000000000000000000000000000
%                                                 Theorem 2.15
%00000000000000000000000000000000000000000000000000000000000000000000000000000000000000000000000000000000000000000000

\begin{teorem}\label{13}
    Let $ L $ be a distributive lattice and $ x \in L $. If $ x \beta_* y $ and there exists a 
    decomposition $ a \oplus b = 1 $ such that $ a \leq x $ and $ b \wedge x \ll L $ then same 
    decomposition exists for $ y \in L $ such that $ a \leq y $ and $ b \wedge y \ll L $
\end{teorem}
\begin{ispat}
    Since $ a \leq x $ and $ b \wedge x \ll L $, it follows from Theorem \ref{11} \ref{11.4} 
    that $ a \beta_* x $. Since $ x \beta_* y $ and $ a \beta_* x $, $ a \beta_* y $. 
    By Theorem \ref{12}, $ a \leq y $ and $ b \wedge y \ll L $.
\end{ispat}

%00000000000000000000000000000000000000000000000000000000000000000000000000000000000000000000000000000000000000000000
%                                                 Theorem 2.16
%00000000000000000000000000000000000000000000000000000000000000000000000000000000000000000000000000000000000000000000

\begin{teorem}

\end{teorem}





  







\end{document}
