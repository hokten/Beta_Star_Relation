\documentclass[a4paper,12pt]{article}
\usepackage[top=3cm,bottom=2cm,left=2cm,right=3cm]{geometry}
%\usepackage{ucs}
\usepackage[utf8]{inputenc}
\usepackage[T1]{fontenc} 
\usepackage{amsmath}
\usepackage{amsfonts}
\usepackage{amsthm}
\usepackage{amssymb}
\usepackage{setspace}
\usepackage{mathptmx}
\usepackage{titlesec}
\usepackage{enumitem}





%============================ Bas ============================
\usepackage[maxnames=7,bibstyle=numeric,backend=bibtex]{biblatex}

%\DeclareNameAlias{default}{first-last/first-last}




\newcommand{\Keywords}[1]{\par\noindent {\small{\textbf{Keywords}\/}: #1}}
\newcommand{\Sinif}[1]{\par\noindent {\small{\textbf{AMS Classification Numbers}\/}: #1}}
\newtheorem{theorem}{Theorem}[section]
\newtheorem{lemma}[theorem]{Lemma}
\newtheorem{remark}[theorem]{Remark}
\newtheorem{proposition}[theorem]{Proposition}
\newtheorem{corollary}[theorem]{Corollary}
\newtheorem{definition}[theorem]{Definition}

\usepackage{filecontents}
\begin{filecontents*}{demo.bib}


  @article{nebiyev,
    author = {Birkenmeier, G. F. and Mutlu, F. T. and Nebiyev, C. and Sökmez, N. and Tercan, A.},
    title   = {Goldie*-Supplemented Modules},
    journal = {Glasgow Mathematical Journal},
    year = 2010,
    volume  = {52A},
    pages   = {41--52}
  }
  @article{refail,
    author = {Alizade, R. And Toksoy, E.},
    title  = {Cofinitely Weak Supplemented Lattices, Indian Journal of Pure},
    volume = {40:5},
    edition = {Second},
    year   = 2009,
    location = {Cambridge},
    pages = {337-346}
  }
  @article{stenstrom,
    author = {Stenstrm, B.},
    title  = {Radicals and socles of lattices},
    journal = {Arch. Math.},
    volume = {XX},
    year   = 1969,
    location = {Cambridge},
    pages = {258-261}
  }
  @book{davey,
    author = {Davey, B. A. and Priestley, H. A.},
    title  = {Introduction to Lattices and Order},
    year  = 2002,
    location = {London},
    publisher = {Cambridge University Press}
  }
  @book{calugeranu,
    author = {C\v{a}lug\v{a}reanu, G.},
    title  = {Lattice Concepts of Module Theory},
    pagetotal = {225},
    year  = 2000,
    location = {London},
    publisher = {Kluwer Academic Publisher}
  }
\end{filecontents*}


\numberwithin{equation}{section}
%\usepackage[lflt]{floatflt}


\renewcommand{\labelenumi}{(\roman{enumi})}
\newcommand{\mycite}[2]{\textnormal{\cite{#1},~#2}}

%\hoffset  = 0.96cm
%\textheight  =614pt
%\footskip  =-40mm
%\marginparwidth = 0mm
%\evensidemargin  =0mm
%\marginparsep  = 0mm
%\topmargin  =30mm  
%\oddsidemargin  =-10.mm    
%\evensidemargin  =0mm    
%\headheight  =0.mm
%\headsep  =0.mm
%\textheight  =220.mm
%\textwidth  =150.mm



\titleformat{\section}{\vspace{.8ex} \bfseries} {\thesection.}{.5em}{}
\bibliography{demo}
\begin{document}
\title{$ \beta_* $ RELATION ON LATTICES}
\author{\begin{tabular}[t]{c}
    Celil NEBİYEV, Hasan Hüseyin ÖKTEN\\ [7mm]
    Department of Mathematics, Ondokuz Mayıs University, \\ Kurupelit-Atakum-Samsun, Turkey \\
    cnebiyev@omu.edu.tr, hokten@gmail.com \\
\end{tabular}}
\date{}
\maketitle
\thispagestyle{empty}
\begin{abstract}
  In this paper, we generalize $ \beta^* $ relation on submodules of a module (see \cite{nebiyev}) to elements of a complete modular lattice. Let $ L $ be 
  a complete modular lattice. We say $ a,b \in L $ are $ \beta_* \ equivalent $, $ a \beta_* b $, if and only if 
  for each $ t \in L $ such that $ a \vee t = 1 $ then $ b \vee t = 1 $ and for each $ k \in L $ such that $ b \vee k = 1 $ then $ a \vee k = 1 $, this is 
  equivalent to $ a \vee b \ll 1/a $ and $ a \vee b \ll 1/b $. 
  We show that the $ \beta_* $ relation is an equivalence relation. Then, we examine $ \beta_* $ relation on weak supplemented lattices. 
  Finally, we show that $ L $ is weakly supplemented if and only if for every $ x \in L $, $ x $ is equivalent to a weak supplement in $ L $. 
  \quad \\
  \quad \\

  \Sinif{06C05, 16D10} \\
  \Keywords{$ \beta_* $-relation, weakly supplemented lattice, complemented lattice, amply supplemented lattice, hollow lattice.
  }
\end{abstract}



%\title{Aggregation According to Classical Kinetics---From
%    Nucleation to Coarsening}
%\author{Hasan mez \\
% \multicolumn{1}{p{.9\textwidth}}{\centering\emph{G. Mill\'an Institute
%  of Fluid Dynamics, Nanoscience and Industrial Mathematics,
%  Universidad Carlos III de Madrid, Spain}} \\
%hokten@gmail.com,nozkan@omu.edu.tr, cnebiyev@omu.edu.tr
%}
\section{Introduction}
Throughout this paper, $ L $ denotes a complete modular lattice with smallest element 0 and 
greatest element 1. A lattice we will mean a complete modular lattice. In a lattice $ L $, an element $ m \in L $ is called $ maximal $ in $ L $ if there is no element between $ m $ and $ 1 $. An element $ a $ of $ L $ 
called $ small $ in $ L $, if $ a \vee b \neq 1 $ holds for every $ b \neq 1 $. This is denoted by $ a \ll L $. An element $ c $ of $ L $ is called a $ supplement $ of $ b $ in $ L $ 
if it is minimal relative to the property $ b \vee c = 1 $. Equivalently, an element $ c $ is a supplement of $ b $ in $ L $ if and only if $ b \vee c = 1 $ and 
$ b \wedge c \ll c/0 $. An element $ c $ of $ L $ is called a $ weak \ supplement $ of $ b $ in $ L $ if $ b \vee c = 1 $ and $ b \wedge c \ll L $. A lattice $ L $ is called 
$ supplemented $ (respectively, $ weakly \ supplemented $) if each element of $ L $ has a supplement (respectively, weak supplement) in $ L $.
For $ a \in L $, we said that $ b \in L $ $ complement $ of $ a $ if $ a \wedge b = 0 $ and $ a \vee b = 1 $ (see \cite{davey}). It is denoted by $ a \oplus b = 1 $ (see \cite{calugeranu}). 
A lattice $ L $ is called $ complemented $ if each element in $ L $ has at least one complemented (see \cite{calugeranu}). A lattice $ L $ is called $ hollow $ if every element with distinct from 
1 small in $ L $. An element $ a $ of $ L $ has $ ample \  supplements $ in $ L $ if for every $ t \in L $ with  $ a \vee t = 1 $, there is a supplement $ t^{'} $ of $ a $ with $ t^{'} \leq t $. 
$ L $ is called $ amply \ supplemented $ if all elements of $ L $ have ample supplements in $ L $. 
In a lattice $ L $, the meet of all maximal (except from 1) elements in $ L $ is called $ radical $ of $ L $, denoted $rad(L) $.  
If $ a \in L $ such that $ a \ll L $ then $ a \leq rad(L) $ (see \mycite{stenstrom}{Proposition 6}). 
For $ a,b \in L $ such that $ a \leq b $, we said that $ b \ lies \ above \ a $ if $ b \ll 1/a $. 
A lattice $ L $ is called $ distributive $ if for any elements $ a,b,c $ of $ L $, $ a \wedge ( b \vee c ) = ( a \wedge b ) \vee ( a \wedge c) $ holds.
\section{$ \beta_* $ Relation}
\begin{definition} \label{1}
  Let $ a,b $ be elements of $ L $. We define a relation $ \beta_* $ on the elements of $ L $ by 
  $ a \beta_* b $ if and only if for each $ t \in L $ such that $ a \vee t = 1 $ then $ b \vee t = 1 $ and for each 
  $ k \in L $ such that $ b \vee k = 1 $ then $ a \vee k = 1 $. 

\end{definition}

\begin{lemma} \label{2}
  $ \beta_* $ is an equivalence relation.
\end{lemma}
\begin{proof}
  The reflexive and symmetric properties are clear. For transitivity, assume $ a \beta_*b $ and $ b \beta_* c $. 
  $ t \in L $ such that $ a \vee t = 1 $. Since $ a \beta_*b $, $ b \vee t = 1 $. So,  by $ b \beta_* c $, 
  $ c \vee t = 1 $. Similarly, for each $ k \in L $ such that $ c \vee k = 1 $ then $ a \vee k=1 $. 
  Finally $ a \beta_* c $.

\end{proof}

\begin{theorem} \label{3}
  Let $ a,b $ be elements of $ L $. Then,
  \begin{enumerate}[label=(\roman{*}), ref=(\roman{*})]

    \item
      $ a \beta_* b $ if and only if $ a \vee c = 1 $ and $ b \vee c = 1 $ for each $ c \in L $ 
      such that $ a \vee b  \vee c = 1 $.  \label{3.1}

    \item
      $ a \beta_* b $ if and only if $ a \vee b \ll 1/a $ and $ a \vee b \ll 1/b $.  \label{3.2}

  \end{enumerate}
\end{theorem}


\begin{proof}
  \begin{enumerate}
    \item
      $ ( \Rightarrow ) $
      Let $ a \beta_* b$ and $ c \in L $ such that $ a \vee b \vee c = 1 $. Since $ a \vee ( b \vee c ) = 1 $ and 
      $ a \beta_* b $, $ b \vee ( b \vee c ) = 1 $. Hence $ b \vee c = 1 $. Similarly, $ a \vee c = 1 $. \\
      $( \Leftarrow )$
      $ t \in L $ such that $ a \vee t = 1 $. 
      Then $ a \vee b \vee t = 1 $. By the hypothesis $ b \vee t = 1 $. 
      Similarly, if $ k \in L $ with $ b \vee k = 1 $ then $ a \vee k = 1 $. So $ a \beta_* b $.
    \item
      $ ( \Rightarrow ) $
      Let $ a \beta_* b $ and $ t \in 1/a $ such that $ a \vee b \vee t = 1 $. 
      Then $ ( a \vee t ) \vee b = 1 $. Since $ a \beta_* b $, $ a \vee t = 1 $. 
      Then $ t=1 $. Therefore $ a \vee b \ll 1/a $. Similarly, $ a \vee b \ll 1/b $. \\
      $( \Leftarrow )$
      Let $ a \vee b \ll 1/a $ , $ a \vee b \ll 1/b $ and $ t \in L $ such that $ a \vee t = 1 $. 
      So $ ( a \vee b ) \vee ( b \vee t ) = a \vee b \vee t = 1 $. Since $ a \vee b \ll 1/b $ and 
      $ b \vee t \in 1/b $, $ b \vee t = 1 $. 
      Similarly, for each $ k \in L $ such that $ b \vee k = 1 $ then $ a \vee k = 1 $. 
  \end{enumerate}
\end{proof}

\begin{theorem} \label{4}
  Let $ a,b $ be elements of $ L $. Then,
  \begin{enumerate}[label=(\roman{*}), ref=(\roman{*})]

    \item
      If $ a \ll L $ and $ a \beta_* b $ then $ b \ll L $.
    \item \label{4.2}
      All small elements in $ L $ is equivalent with $ \beta_* $ equivalence relation.
  \end{enumerate}
\end{theorem}

\begin{proof}
  \begin{enumerate}
    \item
      Let $ a \beta_* b $ , $ a \ll L $ and $ t \in L $ such that $ b \vee t = 1 $. 
      Hence $ a \vee t = 1 $. Since $ a \ll L $, $ t = 1 $. Thus $ b \ll L $. 
    \item
      Let $ a \ll L $ and $ b \ll L $ for $ a,b \in L $. Since $ a \ll L $, if $ a \vee t = 1 $ 
      then $ t = 1 $. Therefore $ b \vee t = 1 $. 
      Similarly, $ a \vee k = 1 $ for each $ k \in L $ such that $ b \vee k = 1 $. 
      Thus $ a \beta_* b $. 
  \end{enumerate}
\end{proof}

\begin{corollary} \label{5}
  $ L $ is hollow if and only if all elements with distinct from $ 1 $ in $ L $ are equivalent with $ \beta_* $ relation. 
\end{corollary}
\begin{proof}
  $ ( \Rightarrow ) $
  Let $ L $ be hollow. Then all elements of $ L $ with distinct from $ 1 $ are small in $ L $. Then by Theorem \ref{4} \ref{4.2}, 
  all elements with distinct from $ 1 $ in $ L $ are equivalent with $ \beta_* $ relation. \\
  $ ( \Leftarrow ) $ 
  Let all elements of $ L $ be equivalent to each other. Let us take any element $ t $ in $ L $ 
  such that $ a \vee t = 1 $. By the hypothesis, $ a \beta_* t $. Hence $ t = t \vee t = 1 $. 
  Therefore $ a \ll L $ and so $ L $ is hollow. 
\end{proof}
\begin{theorem}\label{6}
  Let $ a,b $ be elements of $ L $ such that $ a \leq b $. 
  If $ b $ lies above $ a $ then, $ a \beta_* b $. 
\end{theorem}
\begin{proof}
  Assume $ b $ lies above $ a $. Then, $ b < 1/a $. Since $ a \leq b $ for any $ t \in L $ such that $ a \vee t = 1 $, $ b \vee t = 1 $. 
  Conversely, let $ k \in L $ with $ b \vee k = 1 $. Then $ b \vee a \vee k = 1 $. Since $ b \leq 1/a $ and $ a \vee k \in 1/a $, $ a \vee k = 1 $. Hence $ a \beta_* b $.


\end{proof}
\begin{lemma}\label{7}
  Let $ a,b,c $ be elements of $ L $.
  If $ a \vee b = 1 $ and $ ( a \wedge b ) \vee c =1 $
  then $ a \vee(b \wedge c)=b \vee ( a \wedge c ) = 1 $.
\end{lemma}
\begin{proof}
  Assume $ a \vee b = 1 $ and $ ( a \wedge b ) \vee c = 1 $. Since $ ( a \wedge b ) \vee c = 1 $, $ a = a \wedge 1 = a \wedge \left[ ( a\wedge b ) \vee c \right]= ( a \wedge b ) \vee ( a \wedge c ) $. Then $ 1 = a\vee b $, 
  $ ( a \wedge b ) \vee ( a \wedge c ) \vee b = b \vee ( a \wedge c )$. Similarly $ a \vee ( b \wedge c ) = 1 $.
\end{proof}


\begin{theorem}\label{8}
  Let $ a,b \in L $. If $ a\beta_* b $ then the following conditions hold.
  \begin{enumerate}[label=(\roman{*}), ref=(\roman{*})]
    \item
      If there are exists supplements of $ a $ and $ b $ then these are same. \label{8.1}
    \item
      If there are exists weak supplements of $ a $ and $ b $ then these are same. \label{8.2}
  \end{enumerate}
\end{theorem}
\begin{proof}
  \begin{enumerate}
    \item
      Let $ c $ be a supplement of $ a $. Then $ a \vee c = 1 $. Since $ a \beta_* b $, $ b \vee c=1 $.
      Let $ d \in L $ such that $ d \leq c $ and $ b \vee d = 1 $. Therefore 
      $ a \vee d = 1 $. Since $ c $ is a supplement of $ a $ and $ d \leq c $, $ d = c $. Then 
      $ c $ is a supplement of $ b $. Similarly, interchanging the roles of $ a $ and $ b $ 
      we can show that each supplement of $ b $ is also a supplement of $ a $.
    \item
      Let $ a \beta_* b $ and $ c $ be a weak supplement of $ a $ in $ L $. Therefore $ a \vee c = 1 $ 
      and $ a \wedge c \ll L $. Since $ a \beta_* b $ and $ a \vee c = 1 $, $ b \vee c = 1 $. 
      Let $ t $ be an element of $ L $ such that $ (b \wedge c ) \vee t = 1 $ . By Lemma \ref{7}, 
      $ b \vee ( c \wedge t )=1 $ and since $ a \beta_* b $, $ a \vee ( c \wedge t ) = 1 $. Then by also Lemma \ref{7}, $ ( a \wedge c ) \vee t = 1 $ and since $ a \wedge c \ll L $, $ t = 1 $.
      Therefore $ b \wedge c \ll L $ and so $ c $ is also a weak supplement of $ b $. 
      Similarly, interchanging the roles of $ a $ and $ b $ show that each supplement of $ b $ 
      is also a supplement of $ a $.
  \end{enumerate}
\end{proof}
%00000000000000000000000000000000000000000000000000000000000000000000000000000000000000000000000000000000000000000000
%                                                 Corollary 2.12
%00000000000000000000000000000000000000000000000000000000000000000000000000000000000000000000000000000000000000000000




\begin{theorem}\label{9}
  Let $ L $ be a amply supplemented lattice and $ a,b \in L $. If supplements of $ a $ and $ b $ in $ L $ 
  are the same then $ a \beta_* b $.
\end{theorem}

\begin{proof}
  Let $ t \in L $ such that $ a \vee t = 1 $. Since $ L $ is amply supplemented, there exists 
  a supplement of $ a $ in $ L $ such that $ r \leq t $. By the hypothesis, 
  $ r $ is also a supplement of $ b $. Then $ b \vee r = 1 $. Since $ r \leq t $, $ b \vee t = 1 $. 
  Similarly, interchanging the roles of $ a $ and $ b $ we can show that $ a \vee k = 1 $ for any 
  element $ k $ of $ L $ such that $ b \vee k = 1 $. Therefore $ a \beta_* b $.
\end{proof}
%00000000000000000000000000000000000000000000000000000000000000000000000000000000000000000000000000000000000000000000
%                                                 Corollary 2.12
%00000000000000000000000000000000000000000000000000000000000000000000000000000000000000000000000000000000000000000000

\begin{corollary}\label{10}
  Let $ x,y,c \in L $ such that $ x \leq y $ and $ c $ is a weak supplement of $ x $ in $ L $. Then 
  $ x \beta_* y $ if and only if $ y \wedge c \ll L $.
\end{corollary}

\begin{proof}
  $ ( \Rightarrow ) $: 
  Clear from Theorem \ref{8} \ref{8.2}. \\
  $ ( \Leftarrow ) $:  
  Since $ x \leq y $, for any element $ t $ of $ L $ such that $ x \vee t = 1 $, $ y \vee t = 1 $. 
  Let $ k \in L $ such that $ y \vee k = 1 $. Since $ c $ is weak supplement of $ x $ in $ L $, 
  $ x \vee c = 1 $ and $ x \wedge c \ll L $. Therefore $ y \wedge ( x \vee c ) = 1 \wedge y $, and so 
  $ y = x \vee ( y \wedge c ) $. Hence $ k \vee x \vee ( y \wedge c ) = 1 $. Since $ y \wedge c \ll L $, 
  $ x \vee k = 1 $. Thus $ x \beta_* y $.
\end{proof}

%00000000000000000000000000000000000000000000000000000000000000000000000000000000000000000000000000000000000000000000
%                                                 Theorem 2.13
%00000000000000000000000000000000000000000000000000000000000000000000000000000000000000000000000000000000000000000000

\begin{theorem}\label{11}
  Let $ x,y,z,a,b \in L $ such that $ a \oplus b = 1 $ and $ y $ is a supplement of $ x $ in $ L $. Then
  \begin{enumerate}[label=(\roman{*}), ref=(\roman{*})]
    \item
      If $ z \beta_* y $ then $ z / z \wedge x \cong y / y \wedge x $. \label{11.1}
    \item
      If $ z \beta_* b $ then $ z / z \wedge a \cong b / 0 $. \label{11.2}
    \item
      Let $ z \leq b $. Then $ z \beta_* b $ if and only if $ z = b $.\label{11.3}
    \item
      Let $ b \leq z $. Then $ z \beta_* b $ if and only if $ z \wedge a \ll L $.\label{11.4}
  \end{enumerate}
\end{theorem}
\begin{proof}
  \begin{enumerate}
    \item
      Since $ y $ is a supplement of $ x $ in $ L $ and $ z \beta_* y $, 
      $ x \vee y = x \vee z = 1 $. Since $ z / z  \wedge x \cong z \vee x / x $ and 
      $ y \vee x / x \cong y / y \wedge x $, $ z / z \wedge x \cong y / y \wedge x $. 
      Thus $ z / z \wedge x \cong y / y \wedge x $.
    \item
      By \ref{11.1}, $ z / z \wedge a \cong b / a \wedge b $. Since $ a \oplus b = 1 $, 
      $ z / z \wedge a \cong b / 0 $.
    \item
      $ ( \Rightarrow ) $: 
      Since $ a \oplus b = 1 $ and $ z \beta_* b $, $ a \vee z = 1 $. Also, since $ b $ is a supplement of 
      a in $ L $ and $ z \leq b $, $ z = b $. \\
      $ ( \Leftarrow ) $ 
      Clear from reflexive property of $ \beta_* $.
    \item
      $ ( \Rightarrow ) $:
      Since $ a $ is weak supplement of $ b $ and $ z \beta_* b $, it follows from 
      Theorem \ref{8} \ref{8.2} that $ a $ is also weak supplement of $ z $. Hence $ z \wedge a \ll L $. \\
      $ ( \Leftarrow ) $ 
      It is clear from Corollary \ref{10}.
  \end{enumerate}
\end{proof}

%00000000000000000000000000000000000000000000000000000000000000000000000000000000000000000000000000000000000000000000
%                                                 Theorem 2.14
%00000000000000000000000000000000000000000000000000000000000000000000000000000000000000000000000000000000000000000000

\begin{theorem}\label{12}
  Let $ L $ be a distributive lattice and $ a,b \in L $. If $ a \oplus b = 1 $ and $ a \beta_* x $ 
  then $ a \leq x $ and $ b \wedge x \ll L $.
\end{theorem}
\begin{proof}
  Since $ a \oplus b = 1 $ and $ a \beta_* x $, $ x \vee b = 1 $. Hence $ a \wedge ( x \vee b ) = a \wedge 1 = a $. 
  By the distributive property, $ ( a \wedge x ) \vee ( a \wedge b ) = a $. Since $ a \wedge b = 0 $, 
  $ a \wedge x = a $, and so $ a \leq x $. Also since $ x \beta_* a $ and $ a \leq x $, 
  $ b \wedge x \ll L $ by Theorem \ref{11} \ref{11.4} .
\end{proof}

%00000000000000000000000000000000000000000000000000000000000000000000000000000000000000000000000000000000000000000000
%                                                 Theorem 2.15
%00000000000000000000000000000000000000000000000000000000000000000000000000000000000000000000000000000000000000000000

\begin{theorem}\label{13}
  Let $ L $ be a distributive lattice and $ x \in L $. If $ x \beta_* y $ and there exists a 
  decomposition $ a \oplus b = 1 $ such that $ a \leq x $ and $ b \wedge x \ll L $ then same 
  decomposition exists for $ y \in L $ such that $ a \leq y $ and $ b \wedge y \ll L $
\end{theorem}
\begin{proof}
  Since $ a \leq x $ and $ b \wedge x \ll L $, $ a \beta_* x $ by Theorem \ref{11} \ref{11.4}. 
  Since $ x \beta_* y $ and $ a \beta_* x $, $ a \beta_* y $. 
  By Theorem \ref{12}, $ a \leq y $ and $ b \wedge y \ll L $.
\end{proof}

%00000000000000000000000000000000000000000000000000000000000000000000000000000000000000000000000000000000000000000000
%                                                 Theorem 2.16
%00000000000000000000000000000000000000000000000000000000000000000000000000000000000000000000000000000000000000000000
%
%
%
%
%
%

\begin{theorem}\label{14}
  Let $ x \in L $ and $ k $ be a maximal (except from $ 1 $ ) element of $ L $.
  \begin{enumerate}[label=(\roman{*}), ref=(\roman{*})]
    \item
      If $ a, b \in L $ such that $ a \vee b = 1 $, $ b \neq 1 $ and $ x \beta_* a $ then $ x \not\leq b $. \label{14.1}
      \label{14.1}
    \item
      If $ x \beta_*y $ and $ x \leq k $ then $ y \leq k $. \label{14.2}
    \item
      If $ x \beta_* k $ then $ x \leq k $. \label{14.3}
    \item
      If $ x \beta_* k $ and $ w $ is a weak supplement of $ x $ in $ L $ then $ k = x \vee ( k \wedge x ) $ 
      and  $ k \wedge w \ll L $. \label{14.4}
  \end{enumerate}
\end{theorem}

\begin{proof}
  \begin{enumerate}
    \item
      Assume that $ x \leq b $. Then $ a \vee b \vee x = 1 $. Since $ x \beta_* a $, $ x \vee b = 1 $ so $ b = 1 $. 
            %FIXME BEGIN
      This is a contradiction with $ b $ is distinct from $ 1 $. 
            %FIXME END
      Therefore $ x \not\le b $.
    \item
      Let $ y \not\le k $. Then $ k \vee y = 1 $. Since $ x \beta_* y $, $ k = k \vee x = 1 $. This is a contradiction. 
      Therefore $ y \leq k $.
    \item
      Let $ x \beta_* k $ and $ x \not\le k $. Since $ k $ is a maximal element, $ x \vee k = 1 $. Moreover $ k = 1 $, 
      because $ x \beta_* k $. This is a contradiction. Therefore $ x \leq k $.
    \item
      Let $ x \beta_* k $ and $ w $ is a weak supplement of $ x $ in $ L $. From Theorem \ref{8} \ref{8.2}, 
      also $ w $ is a weak supplement of $ k $. Hence $ k \vee w = 1 $ and $ k \wedge w \ll L $. 
      Since $ x \beta_* k $, $ x \leq k $ by \ref{14.3}. Since $ x \vee w = 1 $ 
      and $ x \leq k $, the modular law yields $ k = x \vee ( k \wedge w ) $.
  \end{enumerate}
\end{proof}

%00000000000000000000000000000000000000000000000000000000000000000000000000000000000000000000000000000000000000000000
%                                                 Theorem 2.17
%00000000000000000000000000000000000000000000000000000000000000000000000000000000000000000000000000000000000000000000

\begin{theorem}\label{15}
    %FIXME BEGIN
  Let $ a, b \in L $ and $ a \oplus b = 1 $. For $ x,s \in a/0 $, if $ x \beta_* s $ in $ L $ then $ x \beta_* s $ in $ a/0 $.
    %FIXME END
\end{theorem}

\begin{proof}
  Let $ k \in L $ such that $ x \vee k = a $. Then $ (x \vee k ) \oplus b = 1 $, and so $ x \vee ( k \vee b ) = 1 $. 
  Hence $ s \vee ( k \vee b ) = 1 $ since $ x \beta_* s $. Then $ \left[ (s \vee k ) \vee b \right] \wedge a = a $, 
  it follows that $ ( s \vee k ) \vee ( b \wedge a ) = a $. We obtain that $ s \vee k = a $. Similarly, interchanging 
  roles of $ x $ and $ s $, we can show that $ x \vee t = a $ for any element $ t $ of $ L $ such that $ s \vee t = a $. 
  Therefore $ x \beta_* s \in a/0 $.
\end{proof}

%00000000000000000000000000000000000000000000000000000000000000000000000000000000000000000000000000000000000000000000
%                                                 Theorem 2.18
%00000000000000000000000000000000000000000000000000000000000000000000000000000000000000000000000000000000000000000000

\begin{theorem}\label{16}
  Let $ x,y,k \in L $ such that $ x \vee k = y \vee k = 1 $, $ k \wedge y \leq k \wedge x $ and $ x \vee y \ll 1/y $. 
  Then $ x \vee y \ll 1/x $.
\end{theorem}

\begin{proof}
  Let $ t \in 1/x $ such that $ ( x \vee y ) \vee t = 1 $. Since $ x \vee k = 1 $, $ t \wedge ( x \vee k ) = t \wedge 1 $, 
  and so $ x \vee ( t \wedge k ) = t $ by the modular law. Hence $ x \vee y \vee ( t \wedge k ) = 1 $. 
  It follows that $ x \vee y \vee \left[ y \vee ( t \wedge k ) \right] = 1 $. Since $ x \vee y \ll 1/y $, 
  $ y \vee ( t \wedge k ) = 1 $. Then by, $ t \vee ( k \wedge y ) = 1 $. 
  Since $ k \wedge y \leq k \wedge x $, $ 1 = t \vee ( k \wedge y ) = t \vee ( k \wedge x ) =t $. Therefore $ x \vee y \ll 1/x $.
\end{proof}
%00000000000000000000000000000000000000000000000000000000000000000000000000000000000000000000000000000000000000000000
%                                                 Theorem 2.19
%00000000000000000000000000000000000000000000000000000000000000000000000000000000000000000000000000000000000000000000

\begin{theorem}\label{17}
  Let $x,y,a,b \in L $. If $ a,b \ll L, x \leq y \vee b $ and $ y \leq x \vee a $ then $ x \beta_* y $.
\end{theorem}

\begin{proof}
  Let $ k \in L $ such that $ x \vee y \vee k = 1 $. Since $ x \leq y \vee b $, $ y \vee b \vee k = 1 $. 
  From $ b \ll L $, $ y \vee k = 1 $. Similarly, $ x \vee k = 1 $. 
  Hence, by Theorem \ref{3} \ref{3.1}, $ x \beta_* y $.
\end{proof}


\begin{theorem}\label{18}
  Let $ x_1,x_2,y_1,y_2 \in L $. If $ x_1 \beta_* y_1 $ and $ x_2 \beta_* y_2 $ then $ ( x_1 \vee x_2 ) \beta_* ( y_1 \vee y_2 ) $.
\end{theorem}

\begin{proof}
  Let $ k \in L $ such that $ (x_1 \vee x_2 ) \vee (y_1 \vee y_2 ) \vee k = 1 $. Since $ x_1 \beta_* x_2 $, 
  $ y_1 \vee x_2 \vee y_2 \vee k = 1 $ and $ x_1 \vee x_2 \vee y_2 \vee k = 1 $. Also since 
  $ x_2 \beta_* y_2 $, $ y_1 \vee y_2 \vee k = 1 $ and $ x_1 \vee x_2 \vee k = 1 $. 
  By Theorem \ref{3} \ref{3.1}, $ ( x_1 \vee x_2 ) \beta_* ( y_1 \vee y_2 ) $.
\end{proof}


\begin{corollary}\label{19}
  Let $ x, y_1, y_2,...,y_n \in L $. If $ x \beta_* y_i $ for $ i=1,2,...,n $ then $ \displaystyle x \beta_* \bigvee_{i=1}^n y_i $.
\end{corollary}
\begin{proof}
  Clear from Theorem \ref{18}.
\end{proof}

\begin{theorem}\label{20}
  Let $ x,y \in L $ and $ j \ll L $. Then $ x \beta_* y $ if and only if $ x \beta_* ( y \vee j ) $.
\end{theorem}

\begin{proof}
  $ ( \Rightarrow ) $ 
  Let $ k \in L $ with $ x \vee k = 1 $. Since $ x \beta_* y $, $ y \vee k = 1 $. Then $ y \vee j \vee k = 1 $. 
  Let $ t \in L $ such that $ ( y \vee j ) \vee t = 1 $. Since $ j \ll L $, $ y \vee t = 1 $. $ x \vee t = 1 $ from $ x \beta_* y $. 
  Hence $ x \beta_* ( y \vee j ) $. \\
  $( \Leftarrow ) $ 
  Let $ k \in L $ with $ x \vee k = 1 $. Since $ x \beta_* ( y \vee j ) $, $ y \vee j \vee k = 1 $. Also, since $ j \ll L $, $ y \vee k = 1 $. 
  Let $ t \in L $ such that $ y \vee t = 1 $. Then $ y \vee j \vee t = 1 $. Since $ x \beta_* ( y \vee j ) $, $ x \vee t = 1 $. Hence $ x \beta_* y $.

\end{proof}

\begin{theorem}\label{21}
  Let $ rad(L)=0 $ and $ a \oplus b = 1 $. If $ x \beta_* a $ for some $ x \in L $ then $ x \oplus b = 1 $. 
\end{theorem}
\begin{proof}
  Since $ a \oplus b = 1 $, $ b $ is a supplement of $ a $. 
  Since $ x \beta_* a $, $ b $ is also a supplement of $ x $ by Theorem \ref{8} \ref{8.1}. 
  Therefore $ b \vee x = 1 $ and 
  $ b \wedge x \ll x/0 $. Since $ rad(L)=0 $, $ x \wedge b \leq rad(L) = 0 $. Hence $ x \oplus b = 1 $
\end{proof}

\begin{theorem}
  $ L $ is weakly supplemented if and only if for every $ x \in L $, $ x $ is equivalent to a weak supplement in $ L $.
\end{theorem}

\begin{proof}
  $ ( \Rightarrow ) $ 
  Let $ x \in L $. Since $ L $ is weakly supplemented, there exists $ z \in L $ such that 
  $ x \vee z = 1 $ and $ x \wedge z \ll L $. Also $ x $ is a weak supplement of $ z $ in $ L $. 
  Since $ \beta_* $ relation is reflexive, $ x \beta_* x $. So, every element of $ L $ is 
  equivalent to a weak supplement element in $ L $. \\
  $ ( \Leftarrow ) $
  Let $ x \in L $. By the hypothesis, there exists a weak supplement $ z \in L $ such that 
  $ x \beta_* z $. Let $ z $ is a weak supplement of $ a $. Thus $ a \vee z = 1 $ and $ a \wedge z \ll L $. 
  Also, $ a $ is a weak supplement element of $ z $ in $ L $. Since $ x \beta_* z $, 
  $ a $ is also weak supplement of $ x $ by Theorem \ref{8} \ref{8.2}.  
\end{proof}
\begin{remark}
  The converse of Theorem \ref{6} is not always true. We can give an example about that. Let $ K $ be a hollow module which is not simple. 
  $ L $ be the lattice of the set of all submodules of $ K \times K $ with respect to the ordering relation of inclusion. 
  Since $ K $ is not simple, $ K $ has a submodule $ T $ with $ T \neq 0 $ and $ T \ll K $. Clearly we see that $ T \times 0 \ll L $ and $ 0 \times T \ll L $ . 
  But $ T \times 0 $ and $ 0 \times T $ don't lie above each other.
\end{remark}


\cleardoublepage
\nocite{*}
\printbibliography[maxnames=99]

\end{document}

